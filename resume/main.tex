\documentclass[a4paper,11pt]{article}

\usepackage[utf8]{inputenc}
\usepackage[T1]{fontenc}
\usepackage[none]{hyphenat}

\usepackage{lmodern}
\usepackage[T1]{fontenc}
\usepackage{helvet}
\renewcommand{\familydefault}{\sfdefault}

% Support langue française
\usepackage[french]{babel}

\usepackage{template/bubblecv}

% Définition d'une couleur bleue professionnelle
\definecolor{cvcolor}{RGB}{41, 128, 185}
\definecolor{cvbordercolor}{RGB}{255, 255, 255}

% Configuration des polices
\renewcommand\cvfont{lmss}
\renewcommand{\familydefault}{\sfdefault}

% Configuration de l'avatar
\renewcommand\cvavatarradius{25}

% Ajustement des espacements
\renewcommand{\cvseparatorspace}{1.5}
\renewcommand{\cvitemspace}{2}

\begin{document}
\begin{cv}[profile][2]
{Farès SIONI}{Chef de Projet \& Développeur Fullstack}

    \cvsection[summary]{Profil}
    Développeur Fullstack et Chef de Projet spécialisé dans la conception d'applications métier complexes.
    Expérience en pilotage de projets stratégiques et gestion d'équipe.
    Expertise en Node.js/TypeScript et Angular avec une approche qualité/performance.
    Passionné par l'innovation technique au service des besoins métier.

\cvsection[work]{Expérience Professionnelle}
    \begin{cvevent}[08/2023][08/2025]
        \cvname{Chef de Projet \& Développeur Fullstack (Alternance)}
        \cvdescription{Enedis Lab, Lyon}
        \begin{itemize}
            \item Pilotage complet d'un projet RH stratégique national : coordination d'une équipe de 3 développeurs, développement fullstack (NestJS, Angular, Prisma) et gestion des parties prenantes (métier, directions régionales).
            \item Architecture et développement incluant l'intégration au SI Enedis (SSO) et mise en place de flux de données automatisés entre applications nationales.
            \item \textbf{Impact :} Transformation d'un processus Excel en système centralisé temps réel, réduction de 50\% du temps de traitement, gain d'au moins une journée de travail mensuelle pour chacune des 25+ directions régionales.
            \item Présentation du projet aux membres du COMEX d'Enedis, démontrant l'importance stratégique de l'application pour l'entreprise.
            \item Gestion de projet selon méthodologies Agiles avec GitLab (issues, MR, milestones) et maintenance évolutive des applications existantes.
        \end{itemize}
    \end{cvevent}

\cvseparator[2]
    \begin{cvevent}[06/2023][08/2023]
        \cvname{Développeur Fullstack (Stage)}
        \cvdescription{AXOPEN, Lyon}
        \begin{itemize}
            \item Optimisation majeure d'une fonctionnalité d'agrégation statistique, réduisant l'empreinte mémoire de 3772MB à 71.4MB (réduction de 98\%), améliorant significativement la gestion des données.
            \item Développement d'une application desktop et de son dashboard web (WPF, .NET, Angular), incluant fonctionnalités sécurisées (authentification, vérification IP).
            \item Réduction de la dette technique via analyse et correction de vulnérabilités (SonarQube), mise à jour des frameworks et refactoring de code.
        \end{itemize}
    \end{cvevent}

\cvsection[education]{Formation}
\begin{cvevent}[2023][2025]
    \cvname{Master Technologies de l'Information et Web}
    \cvdescription{Université Claude Bernard Lyon 1, Lyon}
    \begin{itemize}
        \item Sécurité applicative, AWS Cloud Computing
        \item DevOps \&Cloud : Kubernetes, Terraform, Ansible
        \item Architecture des applications web et systèmes distribués
        \item Intégration, analyse et gestion de grandes masses de données
    \end{itemize}
\end{cvevent}

\begin{cvevent}[2019][2023]
    \cvname{Licence Informatique}
    \cvdescription{Université Claude Bernard Lyon 1, Lyon}
    \textbf{Projet phare :} ColorBurst - Jeu multijoueur temps réel (React, TypeScript, Socket.io) - 18/20, major de promotion
\end{cvevent}

\cvsidebar

\cvsection[contact]{Contact}
\begin{cvitem}[Envelope][4]
    \textbf{Email :}\\
    \href{mailto:pro@fsioni.com}{pro@fsioni.com}
\end{cvitem}

\cvseparator[3]
\begin{cvitem}[Phone][4]
    \textbf{ Téléphone :}\\
    \href{tel:+33695657884}
    {+33 6 95 65 78 84}
\end{cvitem}

\cvseparator[3]
\begin{cvitem}[Github][4]
    \textbf{ GitHub :}\\
    \href{https://github.com/fsioni}{github.com/fsioni}
\end{cvitem}

\cvseparator[3]
\begin{cvitem}[Linkedin][4]
    \textbf{ LinkedIn :}\\
    \href{https://linkedin.com/in/sionifareslor}{linkedin.com/in/sionifareslor}
\end{cvitem}

\cvsection[skills]{Compétences}
    \begin{cvitem}
        \textbf{Technologies Frontend :} \\
        Angular, React, JavaScript / TypeScript, \\
        Interfaces UI/UX, Design Responsive
    \end{cvitem}

    \cvseparator[2]
    \begin{cvitem}
        \textbf{Technologies Backend :} \\
        NestJS, Node.js, Prisma, NextJS, .NET Core, \\
        API RESTful, Flux de données
    \end{cvitem}

    \cvseparator[2]
    \begin{cvitem}
        \textbf{DevOps \& Outils :} \\
        Docker, Git, CI/CD, GitLab, \\
        SonarQube, Monitoring
    \end{cvitem}

    \cvseparator[2]
    \begin{cvitem}
        \textbf{Gestion de Projet :} \\
        Méthodologies Agiles, \\
        Leadership technique, \\
        Communication parties prenantes, \\
        Analyse de besoins métier
    \end{cvitem}

    \cvseparator[2]
    \begin{cvitem}
        \textbf{Base de données :} \\
        PostgreSQL, \\
        Conception de schémas, ORM
    \end{cvitem}

\cvsection[languages]{Langues}
\begin{cvitem}
    \textbf{Français :} Langue maternelle
\end{cvitem}

\cvseparator
\begin{cvitem}
    \textbf{Anglais :} Niveau professionnel
\end{cvitem}

\cvseparator
\begin{cvitem}
    \textbf{Espagnol :} Notions
\end{cvitem}

\cvsection[hobbies]{Centres d'intérêt}
\begin{cvitem}
    Création musicale
\end{cvitem}

\cvseparator
\begin{cvitem}
    Cinéma
\end{cvitem}

\cvseparator
\begin{cvitem}
    Voyage
\end{cvitem}

\cvseparator
\begin{cvitem}
    Running
\end{cvitem}

\end{cv}
\end{document}