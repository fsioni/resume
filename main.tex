\documentclass{article}
\usepackage{template/bubblecv}
\usepackage{lmodern}
\usepackage{xcolor}
\usepackage{helvet}
% Définition d'une couleur bleue professionnelle
\definecolor{cvcolor}{RGB}{41, 128, 185}
\definecolor{cvbordercolor}{RGB}{255, 255, 255}
% Configuration des polices
\renewcommand\cvfont{lmss}
\renewcommand{\familydefault}{\sfdefault}
\renewcommand\cvheadertitlefontscale{2.5}
\renewcommand\cvsectionfontscale{1.8}
\renewcommand{\cveventdatewidth}{20}
% Configuration de l'avatar
\renewcommand\cvavatarradius{25}  % Augmentons légèrement le rayon
% Ajustement des espacements
\renewcommand{\cvseparatorspace}{1.5}
\renewcommand{\cvitemspace}{2}
\begin{document}
\begin{cv}[profile][1.9]
{Farès SIONI}{Chef de Projet \& Développeur Fullstack}

\cvsection[summary]{Profil}
Développeur Fullstack et Chef de Projet avec une expertise en développement d'applications métier complexes. Fort d'une expérience en pilotage d'équipe et gestion de projet stratégique, je combine compétences techniques pointues et capacités de leadership. Passionné par l'amélioration continue et l'innovation technique.

\cvsection[work]{Expérience Professionnelle}
\begin{cvevent}[08/2023][08/2025]
    \cvname{Chef de Projet \& Développeur Fullstack (Alternance)}
    \cvdescription{Enedis Lab, Lyon}
    \begin{itemize}
        \item Direction d'un projet RH stratégique national : pilotage d'une équipe de 3 développeurs, développement fullstack (NestJS, Angular, Prisma) et coordination du déploiement avec les directions régionales.
        \item \textbf{Impact :} Optimisation des processus RH avec un gain d'une journée de travail mensuelle par direction régionale, amélioration significative de la communication interne.
        \newline
        \item Maintenance et évolution (TMA) des applications existantes sur la même stack technique.
    \end{itemize}
\end{cvevent}

\cvseparator[2]
\begin{cvevent}[06/2023][08/2023]
    \cvname{Développeur Fullstack (Stage)}
    \cvdescription{AXOPEN, Lyon}
    \begin{itemize}
        \item Maintenance d'un outil interne (React) et développement d'une application desktop et de son dashboard web (WPF, .NET, Angular).
        \item Réduction significative de la dette technique via mise à jour des frameworks et optimisation du code (SonarQube).
    \end{itemize}
\end{cvevent}

\cvsection[education]{Formation}
\begin{cvevent}[2023][2025]
    \cvname{Master Technologies de l'Information et Web}
    \cvdescription{Université Claude Bernard Lyon 1, Lyon}
    \begin{itemize}
        \item Sécurité applicative, AWS Cloud Computing
        \item DevOps \& Cloud : Kubernetes, Terraform, Ansible
        \item Architecture des applications web et systèmes distribués
        \item Intégration, analyse et gestion de grandes masses de données
    \end{itemize}
\end{cvevent}

\begin{cvevent}[2019][2023]
    \cvname{Licence Informatique}
    \cvdescription{Université Claude Bernard Lyon 1, Lyon}
    \textbf{Projet phare :} ColorBurst - Jeu multijoueur temps réel (React, TypeScript, Socket.io) - 18/20, major de promotion
\end{cvevent}

\cvsidebar

\cvsection[contact]{Contact}
\begin{cvitem}[Envelope][4]
    \textbf{Email :}\\
    \href{mailto:fares.sioni@gmail.com}{fares.sioni@gmail.com}
\end{cvitem}

\cvseparator[3]
\begin{cvitem}[Phone][4]
    \textbf{Téléphone :}\\
    \href{tel:+33695657884}
    {+33 6 95 65 78 84}
\end{cvitem}

\cvseparator[3]
\begin{cvitem}[Globe][4]
    \textbf{GitHub :}\\
    \href{https://github.com/fsioni}{github.com/fsioni}
\end{cvitem}

\cvseparator[3]
\begin{cvitem}[Globe][4]
    \textbf{LinkedIn :}\\
    \href{https://linkedin.com/in/sionifareslor}{linkedin.com/in/sionifareslor}
\end{cvitem}

\cvsection[skills]{Compétences}
\begin{cvitem}
    \textbf{Frontend :} \\
    Angular, React, NextJS, TypeScript
\end{cvitem}

\cvseparator[2]
\begin{cvitem}
    \textbf{Backend :} \\
    NestJS, Prisma, .NET Core
\end{cvitem}

\cvseparator[2]
\begin{cvitem}
    \textbf{DevOps :} \\
    Docker, Git, CI/CD
\end{cvitem}

\cvseparator[2]
\begin{cvitem}
    \textbf{Gestion de Projet :} \\
    Leadership technique, \\
    Communication parties prenantes, \\
    Pair programming
\end{cvitem}

\cvsection[languages]{Langues}
\begin{cvitem}
    \textbf{Français :} Langue maternelle
\end{cvitem}

\cvseparator
\begin{cvitem}
    \textbf{Anglais :} Niveau professionnel
\end{cvitem}

\cvseparator
\begin{cvitem}
    \textbf{Espagnol :} Notions
\end{cvitem}

\cvsection[hobbies]{Centres d'intérêt}
\begin{cvitem}
    Création musicale
\end{cvitem}

\cvseparator
\begin{cvitem}
    Cinéma
\end{cvitem}

\cvseparator
\begin{cvitem}
    Voyage
\end{cvitem}

\end{cv}
\end{document}